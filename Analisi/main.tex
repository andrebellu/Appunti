\documentclass{article}
\usepackage{graphicx}
\graphicspath{ {./img/} }
\usepackage{amsmath}
\usepackage{amsfonts}
\usepackage{amssymb}
\usepackage{mathtools}
\usepackage[dvipsnames]{xcolor}
\usepackage{tkz-euclide}
\usepackage{tikz}
\usepackage{subfiles}
\usepackage{pgfplots}
\usepackage{setspace}
\pgfplotsset{compat=1.15}
\usepackage{mathrsfs}
\usetikzlibrary{arrows}
\usepackage[left=1.5cm, right=1.5cm, top=1.5cm, bottom=1.5cm]{geometry}
\pgfplotsset{compat = newest}
\onehalfspacing % 1.5

\definecolor{ududff}{rgb}{0.30196078431372547,0.30196078431372547,1.}
\definecolor{ccqqqq}{rgb}{0.8,0.,0.}
\definecolor{qqwuqq}{rgb}{0.,0.39215686274509803,0.}


\graphicspath{ {./img/} }

\usepackage{hyperref}
\hypersetup{
    colorlinks,
    citecolor=black,
    filecolor=black,
    linkcolor=black,
    urlcolor=black
}

\renewcommand\fbox{\fcolorbox{red}{white}}

\newcommand{\R}{\mathbb{R}}
\newcommand{\serie}{\sum_{n=1}^{+\infty}}

\title{Analisi I}
\author{Andrea Bellu}
\date{2023/2024}

\begin{document}

\maketitle

\tableofcontents

\subfile{sections/numeri_reali/reali.tex}

\subfile{sections/complementi_numeri_reali/complementi.tex}

\subfile{sections/successioni_limiti/succ_limiti.tex}

\subfile{sections/funzioni/funzioni.tex}

\subfile{sections/funzioni_continue/continue.tex}

\subfile{sections/derivate/derivate.tex}

\subfile{sections/integrali/integrali.tex}

\subfile{sections/serie/serie.tex}

\subfile{sections/differenziali/differenziali.tex}

\end{document}