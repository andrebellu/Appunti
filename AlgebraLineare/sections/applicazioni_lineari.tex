%!TeX root = subfile
\documentclass[../main.tex]{subfiles}
\begin{document}

\section{Applicazioni Lineari}
Quando due insiemi sono dotati di struttura algebrica, tra le applicazioni che
è possibile definire tra essi, assumono particolare significato quelle che
rispettano tale struttura algebrica. Oggetto di questo capitolo sono le
applicazioni tra spazi vettoriali, che conservano la struttura di spazio
vettoriale, dette applicazioni lineari o omomorfismi.\\ Siano $\V$ e $W$ due
spazi vettoriali sul campo $\mathbb{K}$, un'applicazione
\[
    f: \V \rightarrow W
\]
si dice \textbf{lineare} o \textbf{omomorfismo} quando
\begin{enumerate}
    \item $\forall \textbf{v}, \textbf{v'} \in \V, f(\textbf{v}+\textbf{v'}) = f(\textbf{v})+f(\textbf{v'})$
    \item $\forall \textbf{v} \in \V, \forall \alpha \in \mathbb{K}, f(\alpha\textbf{v}) = \alpha f(\textbf{v})$
\end{enumerate}

\subsection{Definizioni}
Nel seguito, se $f: V\to W$ è un'applicazione lineare tra due spazi vettoriali
sullo stesso campo $\mathbb{K}$, diremo che $f$ è un'applicazione lineare
\textbf{iniettiva}, o che $f$ è un \textbf{monomorfismo}, se è iniettiva come
applicazione. Analogamente, diremo che $f$ è un'applicazione lineare
\textbf{suriettiva}, ovvero che è un \textbf{epimorfismo}, se è suriettiva come
applicazione. Un'applicazione lineare che sia biiettiva si dice anche
\textbf{isomorfismo}. Si dice \textbf{endomorfismo} un'applicazione lineare di
uno spazio vettoriale $V$ in sé, \textbf{automorfismo} un endomorfismo
biiettivo.\\ Tra le applicazioni lineari notevoli vi sono:
\begin{itemize}
    \item \textbf{L'applicazione lineare nulla:} $\Theta:V\to W$, che associa a ogni vettore di $V$ il vettore nullo di $W$.
    \item \textbf{L'applicazione identica:} $\iota_v$: $V\to V$, definita da $\iota_V(\textbf{v}) = \textbf{V}\ \forall \textbf{v}\in V$, che è un automorfismo di $V$.
\end{itemize}

\end{document}