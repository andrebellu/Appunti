%!TeX root = subfile
\documentclass[../main.tex]{subfiles}
\begin{document}

\section{Applicazioni Lineari}
Quando due insiemi sono dotati di struttura algebrica, tra le applicazioni che
è possibile definire tra essi, assumono particolare significato quelle che
rispettano tale struttura algebrica. Oggetto di questo capitolo sono le
applicazioni tra spazi vettoriali, che conservano la struttura di spazio
vettoriale, dette applicazioni lineari o omomorfismi.\\ Siano $\V$ e $W$ due
spazi vettoriali sul campo $\mathbb{K}$, un'applicazione
\[
    f: \V \rightarrow W
\]
si dice \textbf{lineare} o \textbf{omomorfismo} quando
\begin{enumerate}
    \item \forall \textbf{v}, \textbf{v'} \in \V, $f(\textbf{v}+\textbf{v'}) = f(\textbf{v})+f(\textbf{v'})$
    \item \forall \textbf{v} \in \V, $\forall \alpha \in \mathbb{K}, f(\alpha\textbf{v}) = \alpha f(\textbf{v})$
\end{enumerate}

\end{document}