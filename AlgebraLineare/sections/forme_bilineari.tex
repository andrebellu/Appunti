%!TeX root = subfile
\documentclass[../main.tex]{subfiles}
\begin{document}

\section{Forme Bilineari e Prodotti Scalari}

% pagine 85 86 87 finisci

\subsection{Forme Bilineari}
Sia $\V$ uno spazio vettoriale sul campo $\mathbb{K}$. Una \textbf{forma
    bilineare} su $\mathbb{V}$ è un'applicazione
\[
    *: \V \times \V \rightarrow \mathbb{K}
\]
tale che $\forall \textbf{v,u,w} \in \mathbb{V} \text{ e } k\in\mathbb{K}$
\begin{enumerate}
    \item $ (\textbf{v}+\textbf{u})* \textbf{w} = (\textbf{v}*\textbf{w})+ (\textbf{u}*\textbf{w})$
    \item $\textbf{v}*(\textbf{u}+\textbf{w}) = (\textbf{v}*\textbf{u})+ (\textbf{v}*\textbf{w})$
    \item $ (k\textbf{v})*\textbf{u}=\textbf{v}* (k\textbf{u})=k (\textbf{v}*\textbf{u})$
\end{enumerate}
Si deduce che $0*\textbf{v} = \textbf{v}*0 = 0, \forall\ \textbf{v}\in\mathbb{V}$.

\subsection{Forma bilineare simmetrica}
Una forma bilineare $*$, su uno spazio vettoriale $\V$, si dice \textbf{forma
    bilineare simmetrica o prodotto scalare} se, comunque si considerino due
vettori $\textbf{v}$ e $\textbf{w}$ in $\V$, si ha:
\[
    \textbf{v}*\textbf{w} = \textbf{w} * \textbf{v}\]

\subsection{Prodotti scalari e ortogonalità}
In uno spazio vettoriale $\V$, con prodotto scalare ''$\cdot$'', due vettori
$\textbf{v}$ e $\textbf{w}$ si dicono \textbf{ortogonali} e si scrive
$\textbf{v}\perp\textbf{w}$ se $\textbf{v}\cdot\textbf{w} = 0$.

\subsection{Complemento ortogonale}
Sia $\V$ uno spazio vettoriale con prodotto scalare ''$\cdot$'' e sia $A$ un
sottoinsieme, non vuoto, di $\mathbb{V}$. Si dice \textbf{complemento
    ortogonale} di $A$ in $\Vx{n}$, l'insieme (si legge $A$ ortogonale)
\[
    A^{\perp} = \textbf{v}\in \mathbb{V}\ |\ \textbf{v}\cdot \textbf{w} = 0,\forall\textbf{w}\in A
\]

\subsubsection{Proposizione}
Sia $\V$ uno spazio vettoriale con prodotto scalare ''$\cdot$'' e sia
$\textbf{w}$ un vettore di $\V$ tale che $\textbf{w}\cdot\textbf{w}\ne0$.
Allora, ogni vettore \textbf{v} di $\V$ si può esprimere come somma di due
vettori $\textbf{w}_1$ e $\textbf{w}_2$, dove $\textbf{w}_1$ è ortogonale a
$\textbf{w}$ e $\textbf{w}_2$ è proporzionale a $\textbf{w}$. \\
\textbf{Dimostrazione:} Ogni vettore $\textbf{v}\in\V$ si può scrivere come:
\[
    \textbf{v} = \left(\textbf{v}-\dfrac{\textbf{v}\cdot \textbf{w}}{\textbf{w}\cdot \textbf{w}}\right) + \left(\dfrac{\textbf{v}\cdot \textbf{w}}{\textbf{w}\cdot \textbf{w}}\textbf{w}\right)
\]
Un calcolo diretto dimostra che $\textbf{w}_1 =
    \textbf{v}-\dfrac{\textbf{v}\cdot \textbf{w}}{\textbf{w}\cdot
        \textbf{w}}\textbf{w}$ è ortogonale $\textbf{w}$ mentre, ovviamente,
$\textbf{w}_2 = \dfrac{\textbf{v}\cdot \textbf{w}}{\textbf{w}\cdot
        \textbf{w}}\textbf{w}$ è proporzionale a $\textbf{w}$, secondo lo scalare
$\dfrac{\textbf{v}\cdot \textbf{w}}{\textbf{w}\cdot \textbf{w}}$

\subsection{Coefficiente di Fourier}
Sia $\V$ uno spazio vettoriale con prodotto scalare ''$\cdot$'' e sia
$\textbf{w}$ un vettore di $\mathbb{V}$ tale che
$\textbf{w}\cdot\textbf{w}\ne0$. Se $\textbf{v}$ è un vettore di $\V$, si dice
\textbf{coefficiente} o \textbf{componente di Fourier} di $\textbf{v}$ lungo
$\textbf{w}$ il numero reale
\[
    \textbf{v}_w = \dfrac{\textbf{v}\cdot \textbf{w}}{\textbf{w}\cdot \textbf{w}}
\]
e si dice \textbf{proiezione} di $\textbf{v}$ su $\textbf{w}$ il vettore $
    \overrightarrow{\textbf{v}} = \textbf{v}_w\textbf{w}$.

\subsection{Forme Quadratiche}
Sia $\V$ uno spazio vettoriale con prodotto scalare ''$\cdot$''. Si dice
\textbf{forma quadratica}, associata al prodotto scalare ''$\cdot$'',
l'applicazione
\[
    q: \V \rightarrow \mathbb{K}
\]
\[
    \textbf{v}\rightarrow\textbf{v}\cdot\textbf{v}
\]

\subsection{Spazi con prodotto scalare definito positivo}
Un prodotto scalare, assegnato in uno spazio vettoriale $\V$ su un campo
ordinato, si dice \textbf{definito positivo} se
$\forall\textbf{v}\in\mathbb{V},\textbf{v}\cdot\textbf{v}\geq0$ e
$\textbf{v}\cdot\textbf{v}=0\iff\textbf{v}=\underbar{0}$

Una forma quadratica si dice \textbf{definita positiva} se tale è il prodotto
scalare cui essa è associata.

\subsection{Norma}
Dato un vettore $\textbf{v}\in\mathbb{V}^{\circ}(\mathbb{R})$ si dice
\textbf{norma} di $\textbf{v}$ il numero reale positivo o nullo
\[
    ||\textbf{v}|| = \sqrt{\textbf{v}\cdot\textbf{v}} = \sqrt{\textbf{v}^2} = \sqrt{q(\textbf{v})}
\]

\subsection{Versore}
Sia $\textbf{v}\ne0$ un vettore di $\mathbb{V}^{\circ} (\mathbb{R})$, si dice
\textbf{versore} di $\textbf{v}$ il vettore
\[
    \textbf{v}' = \dfrac{\textbf{v}}{||\textbf{v}||}
\]

\subsection{Disuguaglianza di Cauchy-Schwarz}
Siano \textbf{v} e \textbf{u} due vettori di $\mathbb{V}^{\circ} (\mathbb{R})$.
Allora
\[
    |\textbf{v}\cdot\textbf{u}|\leq||\textbf{v}||\cdot||\textbf{u}||
\]
ove $|v\cdot u|$ indica il valore assoluto di $\textbf{v}\cdot\textbf{u}$.

\subsubsection{Dimostrazione}

\subsection{Disuguaglianza triangolare}

\subsubsection{Dimostrazione}

\end{document}