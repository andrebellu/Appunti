\documentclass{article}
\usepackage{graphicx}
\graphicspath{ {./img/} }
\usepackage{amsmath}
\usepackage{amsfonts}
\usepackage{amssymb}
\usepackage{mathtools}
\usepackage[dvipsnames]{xcolor}
\usepackage{tkz-euclide}
\usepackage{tikz}
\usepackage{subfiles}
\usepackage{pgfplots}
\usepackage{setspace}
\pgfplotsset{compat=1.15}
\usepackage{mathrsfs}
\usepackage{float}
\usetikzlibrary{arrows}
\usepackage[left=1.5cm, right=1.5cm, top=1.5cm, bottom=1.5cm]{geometry}
\pgfplotsset{compat = newest}
\usepackage[normalem]{ulem}
\UseRawInputEncoding

\onehalfspacing % 1.5

\graphicspath{ {./img/} }

\usepackage{hyperref}
\hypersetup{
    colorlinks,
    citecolor=black,
    filecolor=black,
    linkcolor=black,
    urlcolor=black
}

\renewcommand\fbox{\fcolorbox{red}{white}}

\newcommand{\R}{\mathbb{R}}
\newcommand{\serie}{\sum_{n=1}^{+\infty}}

\title{Fisica Sperimentale}
\author{Andrea Bellu}
\date{AA 2023/2024}

\begin{document}

\maketitle

\tableofcontents
\newpage

\section{Misure}
\subfile{sections/Cinematica/misure.tex}
\section{Variabili Cinematiche}
\subfile{sections/Cinematica/variabili_cinematiche.tex}
\section{Moto}
\subfile{sections/Cinematica/moti_introduzione.tex}

\section{Calcolo vettoriale}
\subfile{sections/Calcolo_vettoriale/calcolo_vettoriale.tex}
\subfile{sections/Cinematica/moti_traiettoria_curvilinea.tex}
\subfile{sections/Cinematica/moto_circolare.tex}

\section{Dinamica del punto}
\subfile{sections/Dinamica/main.tex}


\end{document}